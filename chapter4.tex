\documentclass{article}

\nofiles

\usepackage[a4paper,hmargin=1in,vmargin=1in]{geometry}
\usepackage{amsfonts,amsmath,amsthm,enumitem}

\newenvironment{exercise}[1]
{\noindent \textbf{Exercise #1.}}
{\par \addvspace{3mm}}

\begin{document}

\begin{exercise}{4.2.1}
    Let \(f\) and \(g\) be functions defined on some \(A \subseteq \mathbf{R}\) with \(\lim_{x \to c} f(x) = L\) and
    \(\lim_{x \to c} g(x) = M\) for some limit point \(c\) of \(A\). Then \(\lim_{x \to c} [f(x) + g(x)] = L + M\).

    \begin{enumerate}[label=(\alph*)]
        \item Proof using the Sequential Criterion for Functional Limits and the Algebraic Limit Theorem:
        \begin{proof}
            By the Sequential Criterion for Functional Limits, we know that for any sequence \((x_n) \subseteq A\) with
            \(\lim x_n = c\) and \(x_n \neq c\) for all \(n \in \mathbf{N}\), \(\lim f(x_n) = L\) and
            \(\lim g(x_n) = M\). Then the Algebraic Limit Theorem tells us that \(\lim [f(x_n) + g(x_n)] = L + M\).
            Thus, the Sequential Criterion for Functional Limits implies \(\lim_{x \to c}[f(x) + g(x)] = L + M\).
        \end{proof}

        \item Proof using the \(\epsilon-\delta\) definition of functional limits:
        \begin{proof}
            Let \(\epsilon > 0\) be arbitrary. There exists a \(\delta_0 > 0\) such that for any \(x \in A\), if
            \(0 < |x - c| < \delta_0\), then \(|f(x) - L| < \epsilon/2\). Similarly, there exists a \(\delta_1 > 0\)
            such that for any \(x \in A\), if \(0 < |x - c| < \delta_1\), then \(|g(x) - M| < \epsilon/2\). If
            \(\delta = \min\{\delta_0, \delta_1\}\) and \(0 < |x - c| < \delta\), then
            \(|(f(x) + g(x)) - (L + M)| \leq |f(x) - L| + |g(x) - M| < \epsilon\). Thus,
            \(\lim_{x \to c}[f(x) + g(x)] = L + M\).
        \end{proof}

        \item $\lim_{x \to c}[f(x)g(x)] = LM$.
        \vspace{\baselineskip}
        Proof using the Sequential Criterion for Functional Limits and the Algebraic Limit Theorem:
        \begin{proof}
            By the Sequential Criterion for Functional Limits, we know that for any sequence \((x_n) \subseteq A\) with
            \(\lim x_n = c\) and \(x_n \neq c\) for all \(n \in \mathbf{N}\), \(\lim f(x_n) = L\) and
            \(\lim g(x_n) = M\). Then the Algebraic Limit Theorem tells us that \(\lim [f(x_n)g(x_n)] = LM\). Thus, the
            Sequential Criterion for Functional Limits implies \(\lim_{x \to c}[f(x)g(x)] = LM\).
        \end{proof}

        Proof using the $\epsilon-\delta$ definition of functional limits:
        \begin{proof}
            Let \(\epsilon > 0\) be arbitrary. There exists a \(\delta_0 > 0\) such that for any \(x \in A\), if
            \(0 < |x - c| < \delta_0\), then \(|f(x) - L| < \min\{\frac{\epsilon}{2(|M| + 1)}, 1\}\). Similarly, there
            exists a \(\delta_1 > 0\) such that for any \(x \in A\), if \(0 < |x - c| < \delta_1\), then
            \(|g(x) - M| < \frac{\epsilon}{2(|L| + 1)}\). Then if \(\delta = \min\{\delta_0, \delta_1\}\) and
            \(0 < |x - c| < \delta\),
            \begin{align*}
                |f(x)g(x) - LM| &\leq |M||f(x) - L| + |f(x)||g(x) - M| \\
                                &< |M||f(x) - L| + (|L| + 1)|g(x) - M| \\
                                &< \epsilon.
            \end{align*}
            Thus, \(\lim_{x \to c}[f(x)g(x)] = LM\).
        \end{proof}

    \end{enumerate}
\end{exercise}

\begin{exercise}{4.2.2} For each stated limit, find the largest possible \(\delta\)-neighborhood
that is a proper response to the given \(\epsilon\) challenge.
\begin{enumerate}[label=(\alph*)]
    \item \(\lim_{x \to 3} (5x - 6) = 9\), where \(\epsilon = 1\).
    \[\delta = \frac{1}{5}\]

    \item \(\lim_{x \to 4}\sqrt{x} = 2\), where \(\epsilon = 1\).
    \[\delta = 3\]

    \item \(\lim_{x \to \pi}\lfloor x \rfloor = 3\), where \(\epsilon = 1\).
    \[\delta = \pi - 3\]

    \item \(\lim_{x \to \pi}\lfloor x \rfloor = 3\), where \(\epsilon = 0.01\).
    \[\delta = \pi - 3\]
\end{enumerate}
\end{exercise}

\begin{exercise}{4.2.3}
    % TODO I've written out answers for (a) and (b) but haven't finished a proof for (c) yet.
\end{exercise}

\begin{exercise}{4.3.1}
    Let \(g(x) = \sqrt[3]{x}\).
    \begin{enumerate}[label=(\alph*)]
        \item Prove that \(g\) is continuous at \(c = 0\).
        \begin{proof}
            If \(|x| < \epsilon^3\), then \(|\sqrt[3]{x}| < \epsilon\).
        \end{proof}
    
        \item Prove that \(g\) is continuous at a point \(c \neq 0\).
        \begin{proof}
            Let \(|x - c| < \min{c^{2/3}\epsilon, |c|}\). Then
            \begin{align*}
                |g(x) - g(c)| = |x^{1/3} - c^{1/3}| = \frac{|x - c|}{|x^{2/3} + x^{1/3}c^{1/3} + c^{2/3}|} < 
                \frac{|x - c|}{c^{2/3}} < \epsilon.
            \end{align*}
        \end{proof}
    \end{enumerate}
\end{exercise}

\begin{exercise}{4.3.2}
    Let \(f\) be a function defined on all of \(\mathbf{R}\).
    \begin{enumerate}[label=(\alph*)]
        \item \(f\) is \textit{onetinuous} at \(c\) if 
        \(\forall \epsilon > 0, |x - c| < 1 \implies |f(x) - f(c)| < \epsilon\).
        \medskip \newline
        \(f(x) = k \text{ for some } k \in \mathbf{R} \text{ is \textit{onetinuous}.}\)

        \item \(f\) is \textit{equaltinuous} at \(c\) if 
        \(\forall \epsilon > 0, |x - c| < \epsilon \implies |f(x) - f(c)| < \epsilon\).
        \medskip \newline
        \(f(x) = mx + b \text{ where } m, b \in \mathbf{R} \text{ and } 0 < |m| \leq 1 \text{ is \textit{equaltinuous} 
        and not \textit{onetinuous}.}\)

        \item \(f\) is \textit{lesstinuous} at \(c\) if \(\forall \epsilon > 0, \exists \delta > 0, \delta < \epsilon, 
        |x - c| < \delta \implies |f(x) - f(c)| < \epsilon\).
        \medskip \newline
        \(f(x) = mx + b \text{ where } m, b \in \mathbf{R} \text{ and } |m| < 1 \text{ is \textit{lesstinuous} and not 
        \textit{equaltinuous}}.\)

        \item Is every \textit{lesstinuous} function continuous? Is every continuous function \textit{lesstinuous}?
        \medskip \newline
        \textit{lesstinuous} functions are continuous because there exists a \(\delta > 0\) for every \(\epsilon > 0\). 
        Continuous functions are \textit{lesstinuous} because if there exists a \(\delta > 0\) satisfying a \(\epsilon > 
        0\), then any smaller \(\delta\) also satisfies that \(\epsilon\).
    \end{enumerate}
\end{exercise}

\begin{exercise}{4.3.3}
    Let \(f: A \to \mathbf{R}\) and \(g: B \to \mathbf{R}\) with \(f(A) \subseteq B\) so that \(g \circ f\) is defined
    on \(A\). If \(f\) is continuous at \(c \in A\) and \(g\) is continuous at \(f(c) \in B\), then  \(g \circ f\) is
    continuous at \(c\).
    \begin{enumerate}[label=(\alph*)]
        \item Prove this using the \(\epsilon\)-\(\delta\) characterization of continuity.
        \begin{proof}
            Let \(\epsilon > 0\) be arbitrary. \(g\) is continous at \(f(c)\) so there exists some \(\delta' > 0\) such
            that for all \(y \in B\), if \(\lvert y - f(c) \rvert < \delta'\), then 
            \(\lvert g(y) - g(f(c)) \rvert < \epsilon\). \(f\) is continuous at \(c\) so there exists some 
            \(\delta > 0\) such that for all \(x \in A\), if \(\lvert x - c \rvert < \delta\), then 
            \(\lvert f(x) - f(c) \rvert < \delta' \). This implies that if \(\lvert x - c \rvert < \delta\), then 
            \(\lvert g(f(x)) - g(f(c)) \rvert < \epsilon\). Thus \(g \circ f\) is continuous at \(c\).
        \end{proof}

        \item Prove this using the sequential characterization of continuity.
        \begin{proof}
            \(g\) is continuous at \(f(c)\) so for any sequence \((y_n) \subseteq B\) with \((y_n) \to f(c)\), we have
            \(g(y_n) \to g(f(c))\). Similarly, \(f\)'s continuity at \(c\) implies that for any sequence 
            \((x_n) \subseteq A\) with \((x_n) \to c\), we know \(f(x_n) \to f(c)\). \(f(x_n) \subseteq B\) so \(g\)'s
            continuity implies \(g(f(x_n)) \to g(f(c))\). Thus \(g \circ f\) is continuous at \(c\).
        \end{proof}
    \end{enumerate}
\end{exercise}

\begin{exercise}{4.4.1}
\begin{enumerate}[label=(\alph*)]
    \item Show that $f(x) = x^3$ is continuous on all of $\mathbf{R}$.
    \begin{proof}Given some $\epsilon > 0$, if $|x| < \sqrt[3]{\epsilon}$, then $|x^3| < \epsilon$. Thus $f$ is continuous at $0$. If $c \neq 0$ and $|x - c| < \min\{|c|, \frac{\epsilon}{7c^2}\}$, then $|x| < 2|c|$ so
    \begin{align*}
        |x^3 - c^3| < |x - c||x^2 + xc + c^2| < |x - c|(4c^2 + 2c^2 + c^2) = 7c^2|x - c| < \epsilon
    \end{align*}
    Thus $f$ is continuous at $c \neq 0$.
    \end{proof}
    \item Show that $f$ is not uniformly continuous on $\mathbf{R}$.
    \begin{proof}Let $x_n = n$ and $y_n = n + \frac{1}{n}$. $\lim|x_n - y_n| = \lim|\frac{1}{n}| = 0$ but $|f(x) - f(y)| = 3n + \frac{3}{n} + \frac{1}{n^3} > 3$ so $f$ is not uniformly continuous.
    \end{proof}
    \item Show that $f$ is uniformly continuous on any bounded subset of $\mathbf{R}$.
    \begin{proof}
    Let $A \subset \mathbf{R}$ be bounded. Then there exists some $M > 0$ such that for all $x \in A$, $|x| \leq M$.
    \begin{align*}
        |x^3 - y^3| = |x - y||x^2 + xy + y^2| \leq |x - y|(x^2 + |xy| + y^2) \leq |x - y|3M^2.
    \end{align*}
    If we choose $\delta < \frac{\epsilon}{3M^2}$ and $|x - y| < \delta$, then $|x^3 - y^3| < \epsilon$. Thus, $f$ is uniformly continuous on $A$.
    \end{proof}
\end{enumerate}
\end{exercise}

\begin{exercise}{4.4.2}
    \begin{enumerate}[label=(\alph*)]
        \item Is \(f(x) = \frac{1}{x}\) uniformly continuous on \((0, 1)\)?
        \medskip \newline
        No. Let \(x_n = \frac{1}{n}\) and \(y_n = \frac{1}{n + 1}\). \(|x_n - y_n| = \frac{1}{n^2 + n}\) so 
        \(\lim{|x_n - y_n|} = 0\) but \(|f(x_n) - f(y_n)| = 1\) so by the sequential criterion for the absence of 
        uniform continuity, \(f\) is not uniformly continuous.

        \item Is \(g(x) = \sqrt{x^2 + 1}\) uniformly continuous on \((0, 1)\)?
        \medskip \newline
        Yes. \(g\) is continuous on \([0, 1]\) which is compact so \(g\) is uniformly continuous on \([0, 1]\). 
        \((0, 1)\) is a subset of \([0, 1]\) so \(g\) is uniformly continuous on \((0, 1)\).

        \item Is \(h(x) = x\sin(\frac{1}{x})\) uniformly continuous on \((0, 1)\)?
        \medskip \newline
        Yes. We can extend \(h\) over \([0, 1]\) by letting \(h(0) = 0\). \(h\) is continuous on compact set \([0, 1]\) 
        so \(h\) is uniformly continuous on \([0, 1]\). \((0, 1)\) is a subset of \([0, 1]\) so \(h\) is uniformly 
        continuous on \((0, 1)\).
    \end{enumerate}
\end{exercise}

\begin{exercise}{4.4.3}
    Show that \(f(x) = \frac{1}{x^2}\) is uniformly continuous on the set \([1, \infty)\) but not on the set 
    \((0, \infty)\).
    
    \medskip
    First, we'll show that \(f\) is uniformly continuous on \([1, \infty)\).
    \[\lvert \frac{1}{x^2} - \frac{1}{y^2} \rvert = \lvert \frac{y^2 - x^2}{x^2y^2} \rvert = 
    \frac{(x + y)\lvert x - y \rvert}{x^2y^2} = (\frac{1}{xy^2} + \frac{1}{x^2y}) \lvert x - y \rvert\]
    With \(x, y \in [1, \infty)\), we have \(x, y > 1\) so
    \[\lvert \frac{1}{x^2} - \frac{1}{y^2} \rvert \leq 2\lvert x - y \rvert .\]
    If \(\delta = \frac{\epsilon}{2}\) and \(\lvert x - y \rvert < \delta\), 
    \(\lvert \frac{1}{x^2} - \frac{1}{y^2} \rvert < \epsilon\). Given any \(\epsilon\), the associated \(\delta\) only
    relies on \(\epsilon\) and not the specific values of \(x\) or \(y\). Thus, \(f\) is uniformly continuous on
    \([1, \infty)\).

    \medskip
    Next, we'll show that \(f\) is not uniformly continuous on \((0, 1]\). Let \(x_n = \frac{1}{n}\) and 
    \(y_n = \frac{1}{n + 1}\). \(\lvert x_n - y_n \rvert = \lvert \frac{1}{n(n + 1)} \rvert\) so 
    \(\lim \lvert x_n - y_n \rvert = 0\) but \(\lvert f(x_n) - f(y_n) \rvert = \lvert 2n + 1 \rvert \geq 2\) for all
    \(n \in \mathbf{N}\). By the sequential criterion for the absence of uniform continuity, \(f\) is not uniformly
    continuous on \((0, 1]\).
\end{exercise}

\begin{exercise}{4.5.1} Show how the Intermediate Value Theorem follows as a corollary to the preservation of connected sets.
\begin{proof} Let \(f: [a, b] \to \mathbf{R}\) be a continuous function and \(L\) be a point satisfying either \(f(a) < L < f(b)\) or \(f(a) > L > f(b)\). \([a, b]\) is connected and \(f\) is continuous so \(f([a, b])\) is connected. Theorem 3.4.7 implies that because \(L \in f([a, b])\), there exists some \(c \in [a, b]\) such that \(f(c) = L\).
\end{proof}
\end{exercise}

\begin{exercise}{4.5.2} Provide an example of each of the following, or explain why the request is impossible.
\begin{enumerate}[label=(\alph*)]
    \item A continuous function defined on an open interval with range equal to a closed interval.
    \medskip \newline
    \(f(x) = \sin{x}\) over \((0, 2\pi)\). \(f\) is continuous and has range \([-1, 1]\).

    \item A continuous function defined on a closed interval with range equal to an open interval.
    \medskip \newline
    This is not possible. Closed intervals are compact and continuous functions preserve compactness.

    \item A continuous function defined on an open interval with range equal to an unbounded closed set different from \(\mathbf{R}\).
    \medskip \newline
    \(f(x) = \sec{x}\) over \((-\frac{\pi}{2}, \frac{\pi}{2})\). \(f\) is continuous on this interval and has range \([1, \infty)\).

    \item A continuous function defined on all of \(\mathbf{R}\) with range equal to \(\mathbf{Q}\).
    \medskip \newline
    This is not possible because it violates the Intermediate Value Theorem. Pick any two points \(x, y \in \mathbf{R}\) such that \(x < y\) and \(f(x) \neq f(y)\). The density of the irrationals in the reals implies there exists some \(l \notin \mathbf{Q}\) such that \(l \in (f(x), f(y))\) (or \(l \in (f(y), f(x))\)). The Intermediate Value Theorem implies the existence of some \(z \in (x, y)\) such that \(f(z) = l\). Thus \(f\)'s range is not \(\mathbf{Q}\).
\end{enumerate}
\end{exercise}

\begin{exercise}{4.6.1} Construct a function \(f: \mathbf{R} \to \mathbf{R}\) so that
\begin{enumerate}[label=(\alph*)]
    \item $D_f = \mathbf{Z}^c$.
    \[
        f(x) =
        \begin{cases}
            x, & x \in \mathbf{Q} \\
            \lfloor x + 0.5 \rfloor, & x \notin \mathbf{Q}
        \end{cases}
    \]

    \item \(D_f = \{x : 0 < x \leq 1 \}\).
    \[
        f(x) =
        \begin{cases}
            x, & x \leq 0 \text{ or } x > 1 \text{ or } x \in \mathbf{Q} \\
            0, & 0 < x < 1 \text{ and } x \notin \mathbf{Q}
        \end{cases}
    \]
\end{enumerate}
\end{exercise}

\begin{exercise}{4.6.2}
    Given a countable set \(A = \{a_1, a_2, a_3, ...\}\), define \(f(a_n) = \frac{1}{n}\) and \(f(x) = 0\) for all
    \(x \in A\). Find \(D_f\).

    \medskip \noindent \(D_f = A\).

    \begin{proof}
        For any \(a_n \in A\), let \(\epsilon = \frac{1}{n}\). Any \(V_\delta(a_n)\) is uncountable so there exists some
        \(x \in V_\delta(a_n)\), \(x \notin A\) which implies \(f(x) = 0 \notin V_\epsilon(f(a_n))\). Thus
        \(f\) is not continuous at any \(a_n \in A\).

        For \(c \notin A\), we'll consider the cases when \(c\) is and is not a limit point of \(A\) separately. If
        \(c\) is not a limit point of \(A\), then there exists some \(V_\delta(c)\) such that
        \(V_\delta(c) \cap A = \emptyset\). The for any \(V_\epsilon(f(c))\), we can choose \(\delta \leq \delta_0\) and
        have
        \[x \in V_\delta(c) \implies x \notin A \implies f(x) = 0 \in V_\epsilon(f(c)).\]
        Thus \(f\) is continuous at any point \(c\) that is not a limit point of \(A\).

        If \(c\) is a limit point of \(A\), then given some \(V_\epsilon(f(c))\), choose \(N \in \mathbf{N}\) such that
        \(\frac{1}{N} < \epsilon\). Let \(\delta = \min\{ \lvert c - a_n \rvert \mid a_n \in A \text{ and } n < N \}\).
        We know such a minimum exists because there are a finite number of terms \(a_n\) with \(n < N\). Then for any
        \(x \in V_\delta(c)\), we have \(f(x) < \frac{1}{N}\) so \(f(x) \in V_\epsilon(f(c))\).
    \end{proof}
\end{exercise}

\end{document}