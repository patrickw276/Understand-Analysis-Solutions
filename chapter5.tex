\documentclass{article}

\nofiles

\usepackage[a4paper,hmargin=1in,vmargin=1in]{geometry}
\usepackage{amsfonts,amsmath,amsthm,enumitem}

\newenvironment{exercise}[1]
{\noindent \textbf{Exercise #1.}}
{\par \addvspace{3mm}}

\begin{document}

\begin{exercise}{5.3.2} 
    Let $f$ be differentiable on an interval $A$. If $f'(x) \neq 0$ on $A$, show that $f$ is one-to-one on $A$. Provide an example to show that the converse statement need not be true.

    \begin{proof}Suppose $f$ is not one-to-one on interval $A$. Then there exist two points $x, y \in A$ with $f(x) = f(y)$. By the Mean Value Theorem, there exists some point $z \in (x,y)$ such that
    \[f'(z) = \frac{f(x) - f(y)}{x - y} = 0.\]
    \noindent Thus, if $f'(x) \neq 0$ for all $x \in A$, then $f$ is one-to-one.
    \end{proof}
    
    \noindent As an example that shows the converse is not necessarily true, consider $f(x) = x^2$ on $[0, \infty)$. $f$ is one-to-one and $f'(0) = 0$.
    \end{exercise}
    
    \begin{exercise}{5.3.3} Let $h$ be a differentiable function defined on the interval $[0, 3]$, and assume that $h(0) = 1$, $h(1) = 2$, and $h(3) = 2$.
    
    \begin{enumerate}[label=(\alph*)]
        \item Argue that there exists a point $d \in [0, 3]$ where $h(d) = d$.
        $\mathbf{TODO}$
        
    \end{enumerate}
    \end{exercise}

\end{document}